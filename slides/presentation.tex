% Welcome! This is the unofficial Umeå University template.
% IMPORTANT: 
% This work, "Umeå University Unofficial Beamer Theme", is a derivative
% of "University of Udine Unofficial Beamer Theme" by Marco Basaldella, University of Udine, CC 4.0 BY. 
% (https://www.overleaf.com/latex/templates/university-of-udine-unofficial-beamer-theme/zndkgxrjsdzt) 
%
% In this derived work, changes are made to the colour scheme, title frame, selected fonts as well as 
% the images used in the title-frame and header. Otherwise, the main functionality and commands are 
% all credited to Marco Basaldella (and Till Tantau et al. for creating the beamer document class in
% the first place).  
%
% "Umeå University Unofficial Beamer Theme" is licensed under CC 4.0 by Jesper Erixon.

% See README.md for more full information about this template .

% Note that [usenames,dvipsnames] is MANDATORY due to compatibility
% issues between tikz and xcolor packages.

\documentclass[usenames,dvipsnames,10pt]{beamer} 
% Add option 'aspectratio=169' for 16:9 widescreen 
% Add option  'handout' to ignore animations
% If you have a smaller amount of text, feel free to also try '11pt'! / Jesper

\usepackage[utf8]{inputenc}
\usepackage{verbatim}
\usetheme{umu}
\usepackage{dirtytalk}
\usepackage{minted}

%%% Bibliography
\usepackage[style=authoryear,backend=biber]{biblatex}
\addbibresource{bibliography.bib}

% Author names in publication list are consistent 
% i.e. name1 surname1, name2 surname2
% See https://tex.stackexchange.com/questions/106914/biblatex-does-not-reverse-the-first-and-last-names-of-the-second-author
\DeclareNameAlias{author}{given-family}

%%% Suppress biblatex annoying warning
\usepackage{silence}
\WarningFilter{biblatex}{Patching footnotes failed}

%%% Some useful commands
% pdf-friendly newline in links
\newcommand{\pdfnewline}{\texorpdfstring{\newline}{ }} 
% Fill the vertical space in a slide (to put text at the bottom)
\newcommand{\framefill}{\vskip0pt plus 1filll}

%%% Additional packages, added by Jesper Erixon
% Use babel to neatly translate 'abstract' etc. to swedish  or other supported language
%\usepackage[swedish]{babel}

%%% Enter additional packages below (or above, I can't stop you)! / Jesper
\renewcommand{\proofname}{\sffamily{Proof}}

%%%%%%%%%%%%%%%%%%%%%%%%%%%%%%%%%%%%%%%%%%%%%%%%%%%%%%%%%%%%%%%%%%%%%%%%%%%%%%%%%%%%%
%%%%%%%%%%%%%%%%%%%%%%%%%%%%%%% YOUR PRESENTATION BELOW %%%%%%%%%%%%%%%%%%%%%%%%%%%%%
%%%%%%%%%%%%%%%%%%%%%%%%%%%%%%%%%%%%%%%%%%%%%%%%%%%%%%%%%%%%%%%%%%%%%%%%%%%%%%%%%%%%%
\title[Explorando Dart, Flutter e Angular Dart]{Explorando Dart, Flutter e Angular Dart}
\date[\today]{\small06/11/2020}
\author[Daniel Vieira]{
  Daniel Barboza Vieira\\Bacharel em Engenharia da Computação.
  \pdfnewline
  \texttt{daniel.vieira@sattva.company}
}
\institute{SimTic 2020}

\begin{document}

\begin{frame}
\titlepage
\end{frame}

\begin{frame}{Conteúdo}
\tableofcontents
\end{frame}

\framecard[UmUGreen]{A linguagem Dart}

\framepic{graphics/silver_bullet3}{
	\framefill
    \textcolor{black}{Balas de prata \textbf{não} existem}
    \vskip 0.5cm
}

\framepic[1]{graphics/multi_tool}{
	\framefill
%	\begin{flushright}
%		\textcolor{black}{\textbf{Existem ferramentas\\Que podem ser mais ou menos úteis para uma determinada tarefa}}
%	\end{flushright}
}

\framepic{graphics/dart_intro}{
	\framefill
    %\textcolor{black}{Balas de prata \textbf{não} existem}
    %\vskip 0.5cm
}

\section{Histórico da Linguagem}
\begin{frame}{Histórico da Linguagem}

%\begin{alertblock}{Aviso}
%You can ignore this slide if you're \textbf{not} working with Overleaf.
%\end{alertblock}

%\vskip 0.5cm

\begin{description}

	\item[05/2006] Google disponibiliza o projeto Google Web Toolkit, mais conhecido pela sigla GWT, que é um kit de desenvolvimento para a criação e otimização de aplicações web

\end{description}

\vspace{0.25cm}

O desenvolvimento é feito em Java, e posteriormente o código fonte do projeto é convertido para Javascript, um processo chamado de \emph{transpilação}.

\begin{block}{Objetivos}
Produtividade no desenvolvimento de aplicações de alta performance sem que o desenvolvedor seja expert em Javascript, XMLHttpRequest e as peculiaridades de implementação de cada navegador.
\end{block}

\end{frame}

\begin{frame}{Histórico da Linguagem}

\begin{description}

	\item[10/2011] A linguagem Dart é anunciada na conferência \emph{GOTO} por seus criadores Lars Bak e Kasper Lund.
	\item[11/2013] É disponibilizada a versão 1.0 da linguagem.
	\item[07/2014] Padronização da versão 1.0 da linguagem pela ECMA International - European association for standardizing information and communication systems.
	\item[03/2015] É disponibilizada a versão 1.9 da linguagem, onde abandonou-se a ideia da inclusão de uma máquina virtual Dart nos navegadores em razão de protestos da comunidade.
	\item[03/2015] Fui contratado como desenvolvedor Dart na startup Pindeo Inc., localizada em Beverly Hills, Califórnia.

\end{description}

\end{frame}

\begin{frame}{Histórico da Linguagem}

\begin{description}

	\item[08/2018] A versão 2.0 é lançada, alterando-se o sistema de tipos opcionais para tipos estáticos e inferência de tipos.
	\item[11/2019] A versão 2.6 é lançada, com suporte a compilação nativa para as plataformas Linux,Mac OS e Windows
	\item[12/2019] A versão 2.7 é lançada, com suporte a métodos de extensão, segurança de de tipos nulos (null safety) e interoperabilidade com C.
	\item[03/2019] Fui contratado como desenvolvedor Dart na Element Pré-Pagos, localizada em Barueri-SP.
	
\end{description}

\end{frame}

%\begin{frame}[fragile]
%\frametitle{Compiling}
%
%\begin{alertblock}{Warning}
%You can ignore this slide if you \textbf{are} working with Overleaf.
%\end{alertblock}
%
%To compile this deck you'll need the \texttt{biber} package. Probably your \TeX editor already supports it; if not, you will easily find online the instructions to install it.
%
%\vskip 0.5cm
%
%If you're not using an editor, you can compile this presentation using the command line by running:
%
%\begin{verbatim}
%$ pdflatex main.tex
%$ biber main.bcf
%$ pdflatex main.tex
%$ pdflatex main.tex
%\end{verbatim}


%\end{frame}

\section{Características Gerais}

\begin{frame}[fragile]{Características Gerais}

%For this template we defined four colors, following the graphic profile of Umeå University:
%\begin{itemize}
%\item \textcolor{white}{\marker[UmUBlue]{\texttt{UmUBlue}}}
%\item \textcolor{white}{\marker[UmUGreen]{\texttt{UmUGreen}}}
%\item \textcolor{white}{\marker[UmUPink]{\texttt{UmUPink}}}
%\item \textcolor{white}{\marker[UmUGold]{\texttt{UmUGold}}}
%\end{itemize}
%
%\vskip 0.5cm
%
%You can use these colors as you want in your presentation. For example, you can \textbf{\textcolor{UmUGreen}{color the text in green}} by writing \texttt{\textbackslash textcolor\{UmUGreen\}\{my green text\}}.
%
%\vskip 0.5cm
%
%We also redefined many of the most common \LaTeX{} and Beamer commands, like \texttt{itemize}, \texttt{block}, etc. You will see samples of these commands in the following slides.

\begin{itemize}

	\item{Pertencente a família da linguagem C}
	\item{multiplataforma: web, mobile (iOS e Android), desktop e servidores (Linux,Mac OS e Windows)}	
	\item{produtividade e familiaridade com linguagens já estabelecidas como Java e C\#}
	\item{multiparadigma: funcional, imperativa,orientada a objetos, refletiva}
	\item{Foco na performance de execução e produtividade no desenvolvimento das aplicações do lado cliente}
	\item{Suporte a concorrência através de isolates}
	\item{Integração com C através do pacote dart:FFI}
	\item{Suporte a extensões}
	
\end{itemize}

\end{frame}

\section{Software Development Kit (SDK)}
\begin{frame}{Software Development Kit (SDK)}

\begin{description}

	\item[dart]{As aplicações são executadas por uma máquina virtual com compilação JIT e suporte a snapshots.}
	\item[dart2native]{As aplicações são compiladas e otimizadas para execução em código de máquina (AOT).}
	\item[dart2js]{O código fonte da aplicação é convertido para código fonte javascript otimizado.}
	\item[dartanalyzer]{Analisador de sintaxe (linter).}
	\item[dartaotruntime]{Execução de scripts.}
	\item[dartdevc]{Compilador para desenvolvimento e debugging em navegadores Chrome.}
	\item[dartdoc]{Geração de documentação de código no formato dartdoc.}
	\item[dartfmt]{Formatador de código fonte.}
	\item[pub]{Gerenciador de pacotes}

\end{description}

\end{frame}

\framepic[1]{graphics/pub}{
	\framefill
%	\begin{flushright}
%		\textcolor{black}{\textbf{Existem ferramentas\\Que podem ser mais ou menos úteis para uma determinada tarefa}}
%	\end{flushright}

}

%\begin{minted}{dart}
%
%// Definição de uma função de nível de topo (top level function)
%void imprimeInteiro(int valor) {
%
%  // Imprime o conteúdo da variável numero no console 
%  print('O numero informado em ${DateTime.now()} é $valor.'); 
%
%}
%
%// Ponto de entrada de uma aplicação Dart.
%void main() {
%
%  var numero = 42; // Declaração e inicialização de uma variável.
%  imprimeInteriro(numero); // Chamada de função
%
%}
%  
% \end{minted}

\section{Flutter}

\framepic[1]{graphics/flutter.png}{
	\framefill
%	\begin{flushright}
%		\textcolor{black}{\textbf{Existem ferramentas\\Que podem ser mais ou menos úteis para uma determinada tarefa}}
%	\end{flushright}
}

\begin{frame}

Exemplos de aplicativos desenvolvidos com Flutter:

\begin{enumerate}

	\item NuBank
	\item Rive
	\item Google Stadia
	\item Ebay

\end{enumerate}

\end{frame}

\begin{frame}

\begin{block}{Framework Flutter}
Flutter é um kit de ferramentas de interface do usuário multiplataforma projetado para permitir a reutilização de código em diferentes plataformas e sistemas operacionais enquanto também permite que os aplicativos façam interface diretamente com os serviços da plataforma subjacente. 
\end{block}

Durante o desenvolvimento, os aplicativos Flutter são executados em uma VM que oferece atualização dinâmica (\emph{hot reload}) com estado de mudanças sem a necessidade de uma recompilação completa. Para release, os aplicativos Flutter são compilados diretamente em código de máquina, sejam instruções Intel x64 ou ARM, ou para JavaScript, se direcionado à web. 
\end{frame}

\framepic[1]{graphics/architectural_layout}{
	\framefill
%	\begin{flushright}
%		\textcolor{black}{\textbf{Existem ferramentas\\Que podem ser mais ou menos úteis para uma determinada tarefa}}
%	\end{flushright}

}

%\framesubtitle{Um framework para criação de aplicações multiplataforma com a mesma base} 
%\begin{block}{Goal of the mission}
%Shoot in the Death Star's exhaust port and destroy it before it can fire on the Rebel base.
%\end{block} 
%\begin{alertblock}{Take care!}
%TIE Fighters may chase you while approaching the target.
%\end{alertblock} 
%\begin{exampleblock}{Use the force you must}
%Remember your training with Obi-Wan, and use the Force to make the perfect shot.
%\end{exampleblock} 

\begin{frame}{Flutter}

Para o sistema operacional subjacente, os aplicativos Flutter são empacotados da mesma maneira que qualquer outro aplicativo nativo. Um incorporador específico da plataforma fornece um ponto de entrada; coordena com o sistema operacional subjacente para acesso a serviços como superfícies de renderização, acessibilidade e entrada; e gerencia o loop de eventos da mensagem. O embedder é escrito em uma linguagem apropriada para a plataforma: atualmente Java e C ++ para Android, Objective-C / Objective-C ++ para iOS e macOS e C ++ para Windows e Linux.

\end{frame}

\framecard[UmUBlue]{Demonstração - Flutter}

\end{document}